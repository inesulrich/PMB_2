% Copyright 1994 Peter Williams.
%
% This work may be distributed and/or modified under the
% conditions of the LaTeX Project Public License, either version 1.3
% of this license or (at your option) any later version.
% The latest version of this license is in
%   http://www.latex-project.org/lppl.txt
% and version 1.3 or later is part of all distributions of LaTeX
% version 2005/12/01 or later.
%
% This work has the LPPL maintenance status `maintained'.
% 
% The Current Maintainers of this work are Peter Williams and Thorsten Schnier.
%
% This work consists of all files listed in manifest.txt.
%
% Licence and copyright notice added on behalf of Peter Williams and Thorsten Schnier
% by Clea F. Rees 2009/01/30.
\begin{filecontents}{bibtexlogo.sty}
\def\lowBibTeX{{\reset@font\rmfamily B\kern-.05em%
    \raise.0ex\hbox{\scshape i\kern-.025em b}\kern-.08em%
    T\kern-.1667em\lower.7ex\hbox{E}\kern-.125emX}}
\def\BibTeX{\protect\lowBibTeX}
\end{filecontents}
\documentclass[a4paper]{article}
\usepackage{harvard}
\usepackage{bibtexlogo}

\newcommand{\comname}[1]{{\bf $\backslash$#1}}
\newcommand{\keyword}[1]{{\bf #1}}
\newcommand{\varname}[1]{{\em #1}}
\newcommand{\harvard}{{\sf harvard}}
\newcommand{\Harvard}{{\sf Harvard}}

\title{The \Harvard\ Family of Bibliography Styles}
\author{Peter Williams \\ (pwil3058@bigpond.net.au) \and
Thorsten Schnier \\ (thorsten@archsci.arch.su.edu.au)}

\hyphenation{cite-as-noun poss-ess-ive-cite cit-at-ion-mode cite-year
cite-name bib-lio-gr-aphy-sty-le cit-at-ion-sty-le har-vard-par-en-this
har-vard-year-par-en-this har-vard-item bib-item har-vard-and use-pack-age}

\begin{document}
\bibliographystyle{agsm}
\maketitle
\tableofcontents

\section{Introduction}

This document describes the \harvard\ family of bibliographic styles which
are provided in addition to those described in \citeasnoun{latex} and \citeasnoun{btxdoc}.
This style is primarily intended for use with the \BibTeX\ bibliographic
database management system.
However, provision is also made for hand coding of bibliographies.

\section{Citations}

\subsection{Complete Citations}

There are two primary forms of citation in the \harvard\ style dependent
upon whether the reference is used as a noun or parenthetically.
Additionally, where there are more than two authors, all authors are listed in
the first citation and in subsequent citations just the first author's name
followed by `et al.' is used.
The following example from \citeasnoun{agsm}\ illustrates these points.
\begin{quote}
The major improvement concerns the structure of the interview
(Ulrich~\& Trumbo~1965, p.~112) \ldots .
Later reports (Carlson, Thayer, Mayfield~\& Peterson 1971) record greatly 
increased interviewer reliability for structured interviews.
Wright (1969, p.~408) comments that `undoubtedly interviewer skill is
directly related to the validity, quantity and quality of the interview output',
and this would suggest some sort of interviewer training is called for.
Rowe (1960), for example, found that trained interviewers are better able to
evaluate applicants with some measure of reliability.
In addition Wexley, Sanders~\& Yukl (1973) showed that by extensive interviewer
training all significant contrast effects could be eliminated.
The results of the 1971 study (Carlson et al. 1971) are still relevant, but
efforts to~\ldots.
\end{quote}

To facilitate using a citation as a noun a new command
\comname{citeasnoun} has been created which has the same syntax as the
\comname{cite} command except that multiple citations are {\em not}
permitted.
The effect of this command is that
\begin{verbatim}
As \citeasnoun{btxdoc} and \citeasnoun[Annex~B]{latex} describe \ldots
\end{verbatim}
produces
\begin{quote}
As \citeasnoun{btxdoc} and \citeasnoun[Annex~B]{latex} describe \ldots
\end{quote}
whereas
\begin{verbatim}
The \BibTeX\ \cite{btxdoc} and \LaTeX\ \cite[Annex~B]{latex} manuals \ldots
\end{verbatim}
produces
\begin{quote}
The \BibTeX\ \cite{btxdoc} and \LaTeX\ \cite[Annex~B]{latex} manuals \ldots
\end{quote}
A second new command \comname{possessivecite} is provided for those
instances where it is desired to use the citation as a possessive noun phrase.
This is a variation on the \comname{citeasnoun} command and
multiple citations are not permitted.
As an example of its use
\begin{verbatim}
\possessivecite{latex} description of this feature is \ldots
\end{verbatim}
produces
\begin{quote}
\possessivecite{latex} description of this feature is \ldots
\end{quote}
A third new command \comname{citeaffixed} allows text to be affixed
inside the beginning of the parenthesis of a parenthetical citation.
This command is like the the \comname{cite} command except that it
takes a second argument -- the text to be affixed after the opening 
parenthesis.
For example
\begin{verbatim}
\BibTeX\ manuals \citeaffixed{latex,btxdoc}{e.g.} describe \ldots
\end{verbatim}
produces
\begin{quote}
\BibTeX\ manuals \citeaffixed{latex,btxdoc}{e.g.} describe \ldots
\end{quote}

\subsection{Citation Modes}

By default,
where appropriate, citations are abbreviated automatically after the first
reference when bibliographies are produced by \BibTeX.
Provision is also made for this feature to be accessed during manual coding.
This feature may be overridden by using the \comname{citationmode}
command which takes \keyword{full}, \keyword{abbr} or \keyword{default} as its single
argument.
The command \comname{citationmode\{full\}} makes the system use
full citations, \comname{citationmode\{abbr\}} makes the system use
abbreviated citations and \comname{citationmode\{default\}} causes
the default behaviour of using full citations for the first instance and
abbreviated citations thereafter.

Alternatively, the citation mode may be selected as an
\keyword{full}, \keyword{abbr} or \keyword{default}
option to the
\comname{usepackage} command that invokes the \harvard\ package.
Use of the \keyword{default} option is redundant in that, if no citation
mode option is used, that mode will be selected automatically.

\subsection{Partial Citations}

In addition to the primary forms of citation, the citation commands
\comname{citeyear} and \comname{citename} are provided as building blocks
for more complex citations that authors may (from time to time) require.
\comname{citeyear} behaves like the
\comname{cite} command except that only the year portion of the
citation label is used.
For example,
\begin{verbatim}
\citeyear{btxdoc,latex}
\end{verbatim}
produces \citeyear{btxdoc,latex}.
The parenthesis around the year list may be omitted by modifying the command
name with a single asterisk (e.g. \verb+\citeyear*{btxdoc}+).
\comname{citename} behaves like the
\comname{citeasnoun} command except that only the author name(s)
portion of the citation label is used.
For example,
\begin{verbatim}
\citename{btxdoc}
\end{verbatim}
produces
\begin{quote}
\citename{btxdoc}.
\end{quote}
The use of these commands does not trigger the use of abbreviated citations for
subsequent \comname{citeasnoun} and \comname{cite}
references.

\subsection{Exceptions for Individual Citations}

Occasions arise where an author wishes to override the default behaviour for
an individual citation (e.g. she may wish the citation to use the full list
of authors where the default would use the abbreviated form).
All commands that introduce authors' names into a document (i.e.
\comname{cite}, \comname{citeasnoun}, \comname{citeaffixed},
\comname{possesivcite} and \comname{citename}) may be modified with the
addition of a single asterisk in order to force them to use the full
list of authors names and by a double asterisk to force them to use the
abbreviated form.

\section{Styles}
\subsection{Bibliography Styles}
There are six bibliography styles currently available within the
\harvard\ family, \keyword{agsm} (used in this document) which is based on 
\citeasnoun[pp.~95--98]{agsm}, \keyword{dcu}
which is based upon the conventions in use in the Design Computing Unit,
Department of Architectural and Design Science, University of Sydney,
\keyword{jmr} for the Journal of Management Research,
\keyword{jphysicsB} for the Journal of Physics B,
\keyword{kluwer} which aspires to conform to the requirements of Kluwer Academic
Publishers and \keyword{nederlands} which conforms to Dutch conventions.
They are invoked by the \comname{bibliographystyle} as described in
\citeasnoun[p.~74]{latex} and effect the layout of the entries in the bibliography.

Provided there is no name clash with other \harvard\ options the
bibliography style may be selected by passing it as an option to the
\comname{usepackage} command that invokes the \harvard\ package.

\subsection{Citation Styles}
There are two citation styles currently available within the \harvard\ 
family, \keyword{agsm} (used in this document) and \keyword{dcu} which for the previous
example would produce:
\begin{quote}\citationstyle{dcu}
The \BibTeX\ \cite{btxdoc} and \LaTeX\ \cite[Annex~B]{latex} manuals \ldots
\end{quote}
and for multiple citations such as
\begin{verbatim}
  The original documentation \cite{btxdoc,latex} say \ldots
\end{verbatim}
the \keyword{agsm} citation style produces
\begin{quote}\citationstyle{agsm}
The original documentation \cite{btxdoc,latex} say \ldots
\end{quote}
and the \keyword{dcu} citation style produces
\begin{quote}\citationstyle{dcu}
The original documentation \cite{btxdoc,latex} say \ldots
\end{quote}
The default citation style is \keyword{agsm} and both styles have no effect on the
appearance of the \comname{citeasnoun} citation format.

These styles are invoked by the \comname{citationstyle} command,
for example:
\begin{verbatim}
  \citationstyle{agsm}.
\end{verbatim}
Because these styles affect the format of parenthetical citations, this command 
should appear before any \comname{cite} commands.
Additionally the
citation style may be selected by passing an option to the
\comname{usepackage} command that invokes the \harvard\ package.
In order to avoid name clashes with the \keyword{agsm} and \keyword{dcu}
bibliography styles the options \keyword{agsmcite} and \keyword{dcucite} are
used with the \comname{usepackage} command in order to select \keyword{agsm}
and \keyword{dcu} citation modes respectively.

\subsection{Parenthesis Style}
The type of parenthesis used in citations may be set using the
\comname{harvardparenthesis}
command which takes one argument.
The argument to this command must be one of
\keyword{round}, \keyword{curly}, \keyword{angle}, \keyword{square} or
\keyword{none}.
The default value is \keyword{round}.
If it is a requirement that different parenthesis types are required for
parenthetical cites that for the year portion for a \comname{citeasnoun}
citation then the command \comname{harvardyearparenthesis} may be
used to set the year parenthesis seperately.
This command must be issued after any \comname{harvardparenthesis} command
as that command sets both parenthetical and year parenthesis.
If the bibliographic style chosen is \keyword{agsm} or \keyword{dcu} then the parenthesis
style chosen using \comname{harvardyearparenthesis} is used with the
year portion of the entries in the bibliographic listing.
The options \keyword{round}, \keyword{curly}, \keyword{angle}, \keyword{square}
and \keyword{none} may also be used with the the \comname{usepackage} command
that invokes the \harvard\ package.

Authors of style files for use with the \harvard\ family who wish to make
use of this feature should use the strings \verb+" \harvardleft "+ and
\verb+" \harvardright "+ instead of the respective parenthesis characters where
they wish them to be effected by the selection made with
\comname{harvardparenthesis}.

\subsection{Conjunction Style}
In the previous examples for the \keyword{agsm} bibliographic
style a ``\&'' character is used to signify conjunction between a pair of names
or between the last two names of a list of names.
Similarly the word ``and'' is used for the \keyword{dcu} style.
With these two styles this convention may be overwritten by using
\comname{renewcommand} to redefine the command
\comname{harvardand}.
This should be done after the \comname{citationstyle} command
(if used) as this
command resets it to the default for the style selected.

\section{World Wide Web (WWW) References}

The \keyword{agsm}, \keyword{dcu}, \keyword{jmr}, \keyword{jphysicsB} and
\keyword{kluwer} bibliographic styles support a new bibliographic entry
field \keyword{URL} for specifying the \keyword{URL} of documents that are
available via the World Wide Web.
An example of this is the reference to \possessivecite{latex2htmldoc}
documentation for his \LaTeX 2{\sc html} package in the file
\verb+harvard.bib+ that is enclosed with the source for this document.
When processed by \LaTeX 2{\sc html} documents using the \harvard\ bibliographic
package will have hypertext links created from the citation within the text
to the reference list.
If an entry in the reference list has an \keyword{URL} field then a
hypertext link to the document will be created using the data in that field.

\section{Doing It By Hand}
Hand coding is accomplished much the same as described in \citeasnoun[p.~73]{latex}
except that the new command \comname{harvarditem} is used in place
of \comname{bibitem}.
The syntax of this command is
\begin{quote}
\comname{harvarditem} [\varname{abbr-citation}]\{\varname{full-citation}\}\{\varname{citation-year}\}\{\varname{cite-key}\}
\end{quote}
where
\begin{description}
\item[\varname{abbr-citation}] is the (optional) abbreviated citation
(minus the year) to be used in the text
subsequent to the first mention of a particular reference,
\item[\varname{full-citation}] is the full citation (minus the year)
to be used in the text
on the first mention of a particular reference,
\item[\varname{citation-year}] the year portion of the citation including any
suffices required to disambiguate citations, and
\item[\varname{cite-key}] is the key used in the \comname{cite} and
\comname{citeasnoun} commands.
\end{description}

\section{Acknowledgement}
The motivation for this style came from Fay Sudweeks of the Design Computing
Unit who also originated the formats for the \keyword{dcu} style and proofread
their implementation.

The \keyword{nederlands} bibliographic style was implemented by Werenfried Spit
(spit@vm.ci.uv.es).

The idea for \comname{citeyear} came from Renate Schmidt
(Renate.Schmidt@mpi-sb.mpg.de).

The solution to the mysterious \comname{enddocument} problem came
from Berwin A. Turlach (berwin@core.ucl.ac.be) as did the identification
of a subtle problem with sorting entries in the reference list.
\bibliography{harvard}
\end{document}
